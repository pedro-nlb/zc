\documentclass[A4paper, 12pt, british, reqno]{amsart}

%%% Contents of the preamble
    % Packages ----------------- Line 12
    % General things ----------- Line 72
    % Font definitions --------- Line 87
    % Theorem environments ----- Line 230
    % Tikzcd ------------------- Line 395
    % Author, title, etc ------- Line 416
    % \begin{document} --------- Line 458

%%% Packages

\usepackage{libertine}
\usepackage[libertine]{newtxmath}
% Local font definition; before fontenc, cf. https://tex.stackexchange.com/a/2867

\usepackage[T1]{fontenc}
% This uses 8-bit font encoding (with 256 glyphs) instead of the default 7-bit font encoding (with 128 glyphs). For example, with this option ö is a single glyph in the font, whereas on the 7-bit font encoding the font ö is made by adding an accent to the existing glyph o. A bad consequence of not using this package is that you cannot properly copy-paste such words form the output pdf file. Also, for some reason, funny stuff happens with |, < and > in text.
% Some people suggest to load fontenc before inputenc, most agree that it does not matter.

\usepackage[utf8]{inputenc}
% When you type ä in an editor set up for utf8, the machine stores the character number 228. When TeX reads the file it finds the character number 228 and the macros of inputenc transform this into \"a. Finally fontenc does its thing and transforms this into the command print character 228 (otherwise the two things would be printed separatedly as explained in fontenc).

\usepackage[UKenglish]{babel}
% To manage culturally determined typographical and similar rules, in this case for british english. Some people suggest to load babel after fontenc to avoid warnings, although most agree that it does not matter.

\usepackage{mathtools}
% Loads the amsmath package (\usepackage{amsmath}: miscellaneous improvements such as the commands \DeclareMathOperator and \text). It fixes some quirks it has and adds some useful settings, symbols and environments. It improves the aesthetics as well.

\usepackage{amssymb}
% Extended symbol collection, e.g. \Cap and \Cup. More importantly: the \mathbb command! It loads the amsfonts package (\usepackage{amsfonts}: fraktur letters, bold Greek letters...), so we do not need to include it in the preamble anymore.

\usepackage{mathrsfs}
% Font package (only supports upper case letters).

\usepackage{enumitem}
% To control the layout of enumerate, itemize and description. It supersedes the enumerate package.

\usepackage{tikz-cd}
% To draw commutative diagrams.
\usetikzlibrary{decorations.markings}
% For open and closed immersions.

\usepackage{graphicx}
% An extension of the graphics package, with optional arguments for the \includegraphics command.

\usepackage{todonotes}
% To write to do notes use the command \todo.

\usepackage{xcolor}
% To write in colors.

\usepackage{marginnote}
% To write on margins.

\usepackage{manfnt}
% To draw dangerous bent symbol.

\usepackage{float}
% Improved interface for floating objects such as figures and tables, introducing for example the H modifier to force the position of a float in the page or the boxed float. Should be loaded before hyperref.

\usepackage[backref=page]{hyperref}
% To handle cross-referencing and produce hypertext links in the document. It should be loaded last (with few exceptions), because it redefines many LaTeX commands.
% The backref option inserts links on each bibliography entry to the pages in which the citation was used.
%% The hidelinks option removes colors and boxes around links, but the links remain clickable. On firefox the links are even highlighted when the mouse pointer passes over them.
\renewcommand{\backref}[1]{$\uparrow$~#1}
% Adds an upwards arrow before referencing to the pages in which the citations appear.

\usepackage[noabbrev]{cleveref}
% Enhances cross-referencing features, e.g. to reference to a theorem and automatically include the word theorem.
% No abbreviature option to write figure instead of fig. etc.

%%% General things

% Custom colors
\definecolor{darkgreen}{RGB}{0,75,0}
\definecolor{darkblue}{RGB}{0,0,75}
\definecolor{darkred}{RGB}{75,0,0}
\definecolor{linkred}{rgb}{0.6,0.2,0.2}
\definecolor{linkblue}{rgb}{0,0.2,0.6}
\definecolor{linkgreen}{rgb}{0.2,0.6,0.2}

% Limit table of contents to section titles
\setcounter{tocdepth}{1}

% Sloppy formatting -- often looks better
\sloppy

%%% Font definitions

% Script Font used for sheaves
\DeclareFontFamily{OMS}{rsfs}{\skewchar\font'60}
\DeclareFontShape{OMS}{rsfs}{m}{n}{<-5>rsfs5 <5-7>rsfs7 <7->rsfs10 }{}
\DeclareSymbolFont{rsfs}{OMS}{rsfs}{m}{n}
\DeclareSymbolFontAlphabet{\scr}{rsfs}
\DeclareSymbolFontAlphabet{\scr}{rsfs}

% Sheaves
\newcommand{\sA}{\scr{A}}
\newcommand{\sB}{\scr{B}}
\newcommand{\sC}{\scr{C}}
\newcommand{\sD}{\scr{D}}
\newcommand{\E}{\scr{E}} % Exception (Vector bundles)
\newcommand{\F}{\scr{F}} % Exception (Coherent sheaves)
\newcommand{\G}{\scr{G}} % Exception (Coherent sheaves)
\newcommand{\sH}{\scr{H}}
\renewcommand{\hom}{\scr{H}\negthinspace om} % Exception (Hom-sheaf)
\newcommand{\I}{\scr{I}} % Exception (Ideal sheaves)
\newcommand{\sJ}{\scr{J}}
\newcommand{\sK}{\scr{K}}
\renewcommand{\L}{\scr{L}} % Exception (Line bundles)
\newcommand{\M}{\scr{M}} % Exception (Line bundles)
\newcommand{\sN}{\scr{N}}
\renewcommand{\O}{\scr{O}} % Exception (Structure sheaf)
\newcommand{\sP}{\scr{P}}
\newcommand{\sQ}{\scr{Q}}
\newcommand{\sR}{\scr{R}}
\newcommand{\sS}{\scr{S}}
\newcommand{\sT}{\scr{T}}
\newcommand{\sU}{\scr{U}}
\newcommand{\sV}{\scr{V}}
\newcommand{\sW}{\scr{W}}
\newcommand{\w}{\omega} % Addition (Canonical sheaf)
\newcommand{\sX}{\scr{X}}
\newcommand{\sY}{\scr{Y}}
\newcommand{\sZ}{\scr{Z}}

% Mathcal fonts
\newcommand{\calA}{\mathcal{A}}
\newcommand{\calB}{\mathcal{B}}
\newcommand{\calC}{\mathcal{C}}
\newcommand{\calD}{\mathcal{D}}
\newcommand{\calE}{\mathcal{E}}
\newcommand{\calF}{\mathcal{F}}
\newcommand{\calG}{\mathcal{G}}
\newcommand{\calH}{\mathcal{H}}
\newcommand{\calI}{\mathcal{I}}
\newcommand{\calJ}{\mathcal{J}}
\newcommand{\calK}{\mathcal{K}}
\newcommand{\calL}{\mathcal{L}}
\newcommand{\calM}{\mathcal{M}}
\newcommand{\calN}{\mathcal{N}}
\newcommand{\calO}{\mathcal{O}}
\newcommand{\calP}{\mathcal{P}}
\newcommand{\calQ}{\mathcal{Q}}
\newcommand{\calR}{\mathcal{R}}
\newcommand{\calS}{\mathcal{S}}
\newcommand{\calT}{\mathcal{T}}
\newcommand{\U}{\mathcal{U}} % Exception (Open covers)
\newcommand{\calV}{\mathcal{V}}
\newcommand{\calW}{\mathcal{W}}
\newcommand{\X}{\mathcal{X}} % Exception (Families of varieties)
\newcommand{\Y}{\mathcal{Y}} % Exception (Families of varieties)
\newcommand{\calZ}{\mathcal{Z}}

% Blackboard Bold Symbols
\newcommand{\A}{\mathbb{A}} % Exception (Affine space)
\newcommand{\bbB}{\mathbb{B}}
\newcommand{\C}{\mathbb{C}} % Exception (Complex numbers)
\newcommand{\bbD}{\mathbb{D}}
\newcommand{\bbE}{\mathbb{E}}
\newcommand{\bbF}{\mathbb{F}}
\newcommand{\bbG}{\mathbb{G}}
\newcommand{\Gm}{\mathbb{G}_{\mathrm{m}}} % Addition (Punctured affine line)
\newcommand{\bbH}{\mathbb{H}}
\newcommand{\bbI}{\mathbb{I}}
\newcommand{\bbJ}{\mathbb{J}}
\newcommand{\bbK}{\mathbb{K}}
\newcommand{\bbL}{\mathbb{L}}
\newcommand{\bbM}{\mathbb{M}}
\newcommand{\N}{\mathbb{N}} % Exception (Natural numbers)
\newcommand{\bbO}{\mathbb{O}}
\renewcommand{\P}{\mathbb{P}} % Exception (Projective space)
\newcommand{\Q}{\mathbb{Q}} % Exception (Rational numbers)
\newcommand{\R}{\mathbb{R}} % Exception (Real numbers)
\newcommand{\bbS}{\mathbb{S}}
\newcommand{\bbT}{\mathbb{T}}
\newcommand{\bbU}{\mathbb{U}}
\newcommand{\V}{\mathbb{V}} % Exception (Geometric vector bundle)
\newcommand{\bbW}{\mathbb{W}}
\newcommand{\bbX}{\mathbb{X}}
\newcommand{\bbY}{\mathbb{Y}}
\newcommand{\Z}{\mathbb{Z}} % Exception (Integers)

% Boldfont (categories)
\newcommand{\bfA}{\mathbf{A}}
\newcommand{\Ab}{\mathbf{Ab}}
\newcommand{\bfB}{\mathbf{B}}
\newcommand{\bfC}{\mathbf{C}}
\newcommand{\Cat}{\mathbf{Cat}} % Addition (Categories)
\newcommand{\Coh}{\mathbf{Coh}} % Addition (Coherent sheaves)
\newcommand{\D}{\mathbf{D}} % Exception (Derived category)
\newcommand{\Db}{\mathbf{D}^{\mathrm{b}}} % Addition (Bounded derived category)
\newcommand{\bfE}{\mathbf{E}}
\newcommand{\bfF}{\mathbf{F}}
\newcommand{\bfG}{\mathbf{G}}
\newcommand{\bfH}{\mathbf{H}}
\newcommand{\bfI}{\mathbf{I}}
\newcommand{\bfJ}{\mathbf{J}}
\newcommand{\K}{\mathbf{K}} % Exception (Homotopy category)
\newcommand{\bfL}{\mathbf{L}}
\newcommand{\bfM}{\mathbf{M}}
\newcommand{\Mod}{\mathbf{Mod}} % Addition (Modules)
\newcommand{\bfN}{\mathbf{N}}
\newcommand{\bfO}{\mathbf{O}}
\newcommand{\bfP}{\mathbf{P}}
\newcommand{\PSh}{\mathbf{PSh}} % Addition (Presheaves)
\newcommand{\bfQ}{\mathbf{Q}}
\newcommand{\QCoh}{\mathbf{QCoh}} % Addition (Quasi-coherent sheaves)
\newcommand{\bfR}{\mathbf{R}}
\newcommand{\bfS}{\mathbf{S}}
\newcommand{\Set}{\mathbf{Set}} % Addition (Sets)
\newcommand{\Sh}{\mathbf{Sh}} % Addition (Sheaves)
\newcommand{\bfT}{\mathbf{T}}
\newcommand{\bfU}{\mathbf{U}}
\newcommand{\bfV}{\mathbf{V}}
\renewcommand{\Vec}{\mathbf{Vec}} % Addition (Vector bundles)
\newcommand{\bfW}{\mathbf{W}}
\newcommand{\bfX}{\mathbf{X}}
\newcommand{\bfY}{\mathbf{Y}}
\newcommand{\bfZ}{\mathbf{Z}}

% Mathfrak for ideals
\renewcommand{\a}{\mathfrak{a}}
\renewcommand{\b}{\mathfrak{b}}
\renewcommand{\c}{\mathfrak{c}}
\renewcommand{\d}{\mathfrak{d}}
\newcommand{\e}{\mathfrak{e}}
\newcommand{\m}{\mathfrak{m}}
\newcommand{\n}{\mathfrak{n}}

%%% Theorem environments

% Custom theorem styles (empty fields take default values)
\newtheoremstyle{darkgreentheorem}% name of the style
{}% measure of space to leave above the theorem. E.g.: 3pt
{}% measure of space to leave below the theorem. E.g.: 3pt
{\itshape}% name of font to use in the body of the theorem
{}% measure of space to indent
{\color{darkgreen}\bfseries}% name of head font
{.}% punctuation between head and body
{ }% space after theorem head; " " = normal interword space
{}% Manually specify head
\newtheoremstyle{darkbluedefinition}
{}{}{}{}{\color{darkblue}\bfseries}{.}{ }{}
\newtheoremstyle{darkredexample}
{}{}{}{}{\color{darkred}\bfseries}{.}{ }{}

% Numbered theorems
\theoremstyle{plain}
% \theoremstyle{darkgreentheorem}
\newtheorem{thm}{Theorem}[section]
\newtheorem{lm}[thm]{Lemma}
\newtheorem{prop}[thm]{Proposition}
\newtheorem{cor}[thm]{Corollary}
\newtheorem{conj}[thm]{Conjecture}
\newtheorem{fact}[thm]{Fact}
\theoremstyle{definition}
% \theoremstyle{darkbluedefinition}
\newtheorem{defn}[thm]{Definition}
% \theoremstyle{darkredexample}
\newtheorem{exa}[thm]{Example}
\theoremstyle{remark}
\newtheorem{rem}[thm]{Remark}
\newtheorem{nota}[thm]{Notation}
\newtheorem{q}[thm]{Question}
\newtheorem{exe}[thm]{Exercise}

% Custom numbered theorems
\theoremstyle{plain}
% \theoremstyle{darkgreentheorem}
\newtheorem{innercustomthm}{Theorem}
\newenvironment{cthm}[1]
    {\renewcommand\theinnercustomthm{#1}\innercustomthm}
    {\endinnercustomthm}
\newtheorem{innercustomlm}{Lemma}
\newenvironment{clm}[1]
    {\renewcommand\theinnercustomlm{#1}\innercustomlm}
    {\endinnercustomlm}
\newtheorem{innercustomprop}{Proposition}
\newenvironment{cprop}[1]
    {\renewcommand\theinnercustomprop{#1}\innercustomprop}
    {\endinnercustomprop}
\newtheorem{innercustomcor}{Corollary}
\newenvironment{ccor}[1]
    {\renewcommand\theinnercustomcor{#1}\innercustomcor}
    {\endinnercustomcor}
\newtheorem{innercustomconj}{Conjecture}
\newenvironment{cconj}[1]
    {\renewcommand\theinnercustomconj{#1}\innercustomconj}
    {\endinnercustomconj}
\newtheorem{innercustomfact}{Fact}
\newenvironment{cfact}[1]
    {\renewcommand\theinnercustomfact{#1}\innercustomfact}
    {\endinnercustomfact}
% Definitions
\theoremstyle{definition}
% \theoremstyle{darkbluedefinition}
\newtheorem{innercustomdefn}{Definition}
\newenvironment{cdefn}[1]
    {\renewcommand\theinnercustomdefn{#1}\innercustomdefn}
    {\endinnercustomdefn}
% \theoremstyle{darkredexample}
\newtheorem{innercustomexa}{Example}
\newenvironment{cexa}[1]
    {\renewcommand\theinnercustomexa{#1}\innercustomexa}
    {\endinnercustomexa}
\theoremstyle{remark}
\newtheorem{innercustomrem}{Remark}
\newenvironment{crem}[1]
    {\renewcommand\theinnercustomrem{#1}\innercustomrem}
    {\endinnercustomrem}
\newtheorem{innercustomnota}{Notation}
\newenvironment{cnota}[1]
    {\renewcommand\theinnercustomnota{#1}\innercustomnota}
    {\endinnercustomnota}
\newtheorem{innercustomq}{Question}
\newenvironment{cq}[1]
    {\renewcommand\theinnercustomq{#1}\innercustomq}
    {\endinnercustomq}
\newtheorem{innercustomexe}{Exercise}
\newenvironment{cexe}[1]
    {\renewcommand\theinnercustomexe{#1}\innercustomexe}
    {\endinnercustomexe}

% Unnumbered theorems
\theoremstyle{plain}
% \theoremstyle{darkgreentheorem}
\newtheorem*{uthm}{Theorem}
\newtheorem*{ulm}{Lemma}
\newtheorem*{uprop}{Proposition}
\newtheorem*{ucor}{Corollary}
\newtheorem*{uconj}{Conjecture}
\newtheorem*{ufact}{Fact}
\theoremstyle{definition}
% \theoremstyle{darkbluedefinition}
\newtheorem*{udefn}{Definition}
% \theoremstyle{darkredexample}
\newtheorem*{uexa}{Example}
\theoremstyle{remark}
\newtheorem*{urem}{Remark}
\newtheorem*{unota}{Notation}
\newtheorem*{uq}{Question}
\newtheorem*{uexe}{Exercise}

% Cross-referencing
\crefname{thm}{theorem}{theorems}
\Crefname{thm}{Theorem}{Theorems}
\crefname{lm}{lemma}{lemmas}
\Crefname{lm}{Lemma}{Lemmas}
\crefname{prop}{proposition}{propositions}
\Crefname{prop}{Proposition}{Propositions}
\crefname{cor}{corollary}{corollaries}
\Crefname{cor}{Corollary}{Corollaries}
\crefname{conj}{conjecture}{conjectures}
\Crefname{conj}{Conjecture}{Conjectures}
\crefname{fact}{fact}{facts}
\Crefname{fact}{Fact}{Facts}
\crefname{defn}{definition}{definitions}
\Crefname{defn}{Definition}{Definitions}
\crefname{exa}{example}{examples}
\Crefname{exa}{Example}{Examples}
\crefname{rem}{remark}{remarks}
\Crefname{rem}{Remark}{Remarks}
\crefname{nota}{notation}{notations}
\Crefname{nota}{Notation}{Notations}
\crefname{q}{question}{questions}
\Crefname{q}{Question}{Questions}
\crefname{exe}{exercise}{exercises}
\Crefname{exe}{Exercise}{Exercises}
% More cross-referencing
\crefname{cthm}{theorem}{theorems}
\Crefname{cthm}{Theorem}{Theorems}
\crefname{clm}{lemma}{lemmas}
\Crefname{clm}{Lemma}{Lemmas}
\crefname{cprop}{proposition}{propositions}
\Crefname{cprop}{Proposition}{Propositions}
\crefname{ccor}{corollary}{corollaries}
\Crefname{ccor}{Corollary}{Corollaries}
\crefname{cconj}{conjecture}{conjectures}
\Crefname{cconj}{Conjecture}{Conjectures}
\crefname{cfact}{fact}{facts}
\Crefname{cfact}{Fact}{Facts}
\crefname{cdefn}{definition}{definitions}
\Crefname{cdefn}{Definition}{Definitions}
\crefname{cexa}{example}{examples}
\Crefname{cexa}{Example}{Examples}
\crefname{crem}{remark}{remarks}
\Crefname{crem}{Remark}{Remarks}
\crefname{cnota}{notation}{notations}
\Crefname{cnota}{Notation}{Notations}
\crefname{cq}{question}{questions}
\Crefname{cq}{Question}{Questions}
\crefname{cexe}{exercise}{exercises}
\Crefname{cexe}{Exercise}{Exercises}

%%% Tikzcd

% Open and closed immersion arrows.
\makeatletter
\tikzcdset{
open/.code={\tikzcdset{hook, circled};},
closed/.code={\tikzcdset{hook, slashed};},
circled/.code={\tikzcdset{markwith={\draw (0,0) circle (.375ex);}};},
slashed/.code={\tikzcdset{markwith={\draw[-] (-.4ex,-.4ex) -- (.4ex,.4ex);}};},
markwith/.code={
\pgfutil@ifundefined{tikz@library@decorations.markings@loaded}%
{\pgfutil@packageerror{tikz-cd}{You need to say %
\string\usetikzlibrary{decorations.markings} to use arrow with markings}{}}{}%
\pgfkeysalso{/tikz/postaction={/tikz/decorate,
/tikz/decoration={
markings,
mark = at position 0.5 with
{#1}}}}},
}
\makeatother

\usepackage{tikz}

\newcounter{mybox}
\newcommand\tikzmark[1]{%
\tikz[remember picture,overlay] \node[inner xsep=0pt] (#1) {};
}
\newcommand\ColorBox[2][]{%
\stepcounter{mybox}%
\node[draw=red!70!black,fill=red!20,align=left,#1] (box\themybox) {#2};
}

%%% Author, title, etc.

% Author info
\author{Pedro N\'{u}\~{n}ez}
\address{Pedro N\'{u}\~{n}ez \newline
\indent Albert-Ludwigs-Universit\"{a}t Freiburg, Mathematisches Institut \newline
\indent Ernst-Zermelo-Straße 1, 79104 Freiburg im Breisgau (Germany)}
\email{\href{mailto:pedro.nunez@math.uni-freiburg.de}{pedro.nunez@math.uni-freiburg.de}}
\renewcommand*{\urladdrname}{\itshape Homepage}
\urladdr{\href{https://home.mathematik.uni-freiburg.de/nunez/}{https://home.mathematik.uni-freiburg.de/nunez}}
\thanks{The author gratefully acknowledges support by the DFG-Graduiertenkolleg GK1821 ``Cohomological Methods in Geometry'' at the University of Freiburg.}

% Content details
%\keywords{...}
%\subjclass[...]{...}
\title[Zariski Cancellation]{Zariski Cancellation}
\date{4th November 2020}

% Links and pdf options
\makeatletter
\hypersetup{
  pdfauthor={\authors},
  pdftitle={\@title},
  %pdfsubject={\@subjclass},
  %pdfkeywords={\@keywords},
  pdfstartview={Fit},
  pdfpagelayout={TwoColumnRight},
  pdfpagemode={UseOutlines},
  bookmarks,
  colorlinks,
  linkcolor=linkblue,
  citecolor=linkgreen,
  urlcolor=linkred}
\makeatother

\tikzset{
    symbol/.style={
	draw=none,
	every to/.append style={
	    edge node={node [sloped, allow upside down, auto=false]{$#1$}}}
    }
}

% Math operators
\DeclareMathOperator{\Hom}{Hom}

% Other commands
\newcommand{\ot}{\otimes}
\newcommand{\op}{\oplus}

\begin{document}

%%% Contents of the document
    % First section ---------------- Line 468

\maketitle

\begin{abstract}
    Following \cite{hoc72} we provide an example of rings\footnote{We only consider the category of commutative rings with $1$.} $B,C$ such that $B\not\cong C$ but $B[t]\cong C[t]$, where $t$ is an indeterminate.
    As a preparation for this counterexample we also study the notion of projective module and the hairy ball theorem.	
\end{abstract}

\tableofcontents

\begin{center}
    \textcolor{gray}{---parts in gray will be omitted during the talk---}
\end{center}

\section{Introduction}

Let $R$ be a ring.
We can form the polynomial ring $R[t]$ in one variable $t$ with coefficients in $R$.
This construction is functorial, and hence
\[ R\cong S \Rightarrow R[t]\cong S[t]. \]
The goal of this talk is to show with an explicit counterexample due to Hochster \cite{hoc72} that the converse is not true.

In the process of constructing this counterexample we will come across a projective module which, as a consequence of the hairy ball theorem, is not a free module.
Therefore we will discuss projective modules and the hairy ball theorem before jumping into the counterexample.

\section{Projective modules}

\begin{defn}
    Let $\calC$ be a category.
    An object $P\in \calC$ is called \textit{projective} if the following lifting problem can always be solved:
    
    \begin{center}
	\begin{tikzcd}
	    & M\arrow[twoheadrightarrow]{d}{\text{epi}} \\
	    P\arrow{r}\arrow[dashed]{ur} & N
	\end{tikzcd}
    \end{center}
\end{defn}

\begin{lm}
    Let $\calA$ be an abelian category and let $P\in \calA$ be an object.
    The following are equivalent:
    \begin{enumerate}
	\item $P$ is projective.
	\item $\Hom_{\calA}(P,-)$ is exact.
	\item Every short exact sequence of the form
	    \[ 0 \to N\to M\to P\to 0 \]
	    splits.
    \end{enumerate}
    \begin{proof}
	We start with $(1)\Rightarrow (2)$.
	Assume $P$ is projective and consider a short exact sequence
	\[ 0 \to A\to B\to C\to 0. \]
	Since $\Hom_{\calA}(P,-)$ is always left exact, we only need to show that the induced map $\Hom_{\calA}(P,B)\to \Hom_{\calA}(P,C)$ is surjective.
	But $B\to C$ is an epimorphism, so this is precisely what $P$ being projective means by definition.

	Next we show $(2)\Rightarrow (3)$.
	Assume $\Hom_{\calA}(P,-)$ is exact and consider a short exact sequence
	\[ 0\to N\to M\to P\to 0. \]
	Applying $\Hom_{\calA}(P,-)$ we get a surjection $\Hom_{\calA}(P,M)\to \Hom_{\calA}(P,P)$, and the identity on $P$ comes then from the desired section $\sigma\colon P\to M$.

	Let us check finally that $(3)\Rightarrow (1)$.
	We are given the following situation:
	\begin{center}
	    \begin{tikzcd}
		 & M\arrow[twoheadrightarrow]{d}{\text{epi}} \\
		 P\arrow{r} & N
	    \end{tikzcd}
	\end{center}
	All finite limits exist in $\calA$, so we may consider the cartesian square
	\begin{center}
	    \begin{tikzcd}
		P\times_{N} M\arrow[swap]{d}{f}\arrow{r}{g} & M\arrow[twoheadrightarrow]{d}{\text{epi}} \\
		P\arrow{r} & N
	    \end{tikzcd}
	\end{center}
	Epimorphisms are stable under pullback in abelian categories, so $f$ is also an epimorphism.
	By assumption, we can find a section $\sigma\colon P\to P\times_{N}M$ splitting the corresponding short exact sequence.
	The composition $g\circ \sigma\colon P\to M$ is then the desired lift.
    \end{proof}
\end{lm}

Let us look now at the abelian category of modules over a ring $R$.
What does it mean for an $R$-module to be projective?

\vspace{8mm}
\begin{tikzpicture}[remember picture,overlay]
    \ColorBox[xshift=23mm,fill=red!30,draw=red]{non ppal. ideal \\ \hspace{4mm}over a DD \\ \hspace{11.5mm}$\neq$ }
    \ColorBox[xshift=57mm,fill=red!30,draw=red]{\hspace{4mm}$\Q$ \\ over $\Z$ \\ \hspace{4.5mm}$\neq$ }
    \ColorBox[xshift=83mm,fill=red!30,draw=red]{\hspace{5mm}$(x,y)$ \\ over $\C[x,y]$ \\ \hspace{7mm}$\neq$ }
\end{tikzpicture}
\vspace{10mm}
\begin{center}
    \begin{tikzcd}
	\{ \text{Free} \} \tikzmark{a}\arrow[symbol=\subseteq]{r} & \{ \text{Projective} \} \tikzmark{b}\arrow[symbol=\subseteq]{r} & \{ \text{Flat} \} \arrow[symbol=\subseteq]{r} & \{ \text{Torsion-free} \}
    \end{tikzcd}
\end{center}
\vspace{6mm}
\begin{tikzpicture}[remember picture,overlay]
    \ColorBox[xshift=27.5mm,fill=green!30,draw=green]{\hspace{8.5mm} = \\ / PID or local}
    \ColorBox[xshift=60mm,fill=green!30,draw=green]{\hspace{10mm} = \\ f.g. / Noeth.}
    \ColorBox[xshift=86mm,fill=green!30,draw=green]{\hspace{2mm} = \\ / DD}
\end{tikzpicture}
\vspace{5mm}


\bibliographystyle{alpha}
\bibliography{main}
\vfill

\end{document}

